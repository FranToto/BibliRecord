\documentclass[]{report}

\usepackage{hyperref}
\usepackage{graphicx}
\usepackage{setspace}
\usepackage{geometry}
\usepackage{amsmath}
\usepackage{amssymb}
\usepackage{mathrsfs}
\usepackage{gensymb}
\usepackage{color}
\usepackage{enumerate}
\usepackage{natbib}

\begin{document}
	
	\tableofcontents
	
	\chapter{Bibliography Record}
	\section{On regime Shift and Ecological collapse}
	\begin{itemize}
		\subsection{Theory and Generalities on Regime Shift and resilience}
		
		\item \cite{scheffer2001catastrophic} : Catastrophic regime shifts in ecosystems.
		\\First, the contrast among states in ecosystems is usually due to a shift in dominance among organisms with different life forms.
		Second, state shifts are usually triggered by obvious stochastic events such as pathogen outbreaks, fires or climatic extremes.
		Third, feedbacks that stabilize different states involve both biological and physical and chemical mechanisms.
		\\The main implication of the insights presented here is that efforts to reduce the risk of unwanted state shifts should address the gradual changes that affect resilience rather than merely control disturbance.
		\\ The challenge is to sustain a large stability domain rather than to 	control fluctuations. Stability domains typically depend on slowly changing variables such as land use, nutrient stocks, soil
		properties and biomass of long-lived organisms.
		\\These factors may be predicted, monitored and modified. In contrast, stochastic events that trigger state shifts (such as hurricanes, droughts or disease outbreaks) are usually difficult to predict or control.
		\\\textbf{Relevancy for PhD} : Very High. One of those bible for explaining theory behind R-S.
		\item \cite{dakos2015resilience} : Resilience indicators: prospects and limitations for early warnings of regime shifts.
		\\\textbf{Relevancy for PhD} : Very High. One of those bible for explaining mismatch between theory Early Warning Signals and practical use because of data quality.
		\item \cite{scheffer2012anticipating} : Anticipating Critical Transitions.
		\\\textbf{Relevancy for PhD} : Very High. One of those bible for explaining theory behind R-S and Early Warning Signals.
		\item \cite{scheffer2009early} : Early-warning signals for critical transitions.
		\\\textbf{Relevancy for PhD} : Very High. One of those bible for explaining theory behind R-S and Early Warning Signals.
		\item \cite{thrush2009forecasting} : Forecasting the limits of resilience : integrating empirical research and theory.
		\\ Coastal zone : Some of the most valuable ecosystems. Current decline in these ecosystems occurs across a broad range of spatial and temporal scales but rapid and broad scale changes (R-S) are increasingly being described. Such changes can be predicted (with some level oc accuracy) when single environmental or biological drivers are sufficiently strong to force an ecological system into on alternative state. $\rightarrow$ Role of light and remote sensing method in macroalgae key species, major topic of my PhD.
		\\Good review of R-S in applied ecology.
		\\Major barrier to forecasting shifts in resilience of these ecosystems is the disparity between theory and what is practically testable and measurable.
		\\ Review of recent studies to gain insight into the ecological mechanisms underpinning resilience and suggest designs for field experiments.
		\\ Most likely to detect R-S because of the loss of specific species or group (functionally important). Our ability to detect shifts will depend on sampling opportunities over different scales for different ecosystems.
		\\Measuring resilience requires determining whether there are threshold that separate different stability domains, and at present only way to detect a threshold is to cross it. However, we should be able to identify signs of shifts.
		\\ The key issue is defining how close a system is to a threshold and what can we actually measure in natural ecosystems to better understand resilience.
		\\ While resilience and regime shifts are temporal concepts, thresholds can occurs spatially.
		\\Examples of species playing major roles in influencing habitat, recovery rates or energy transfer can affect resilience.
		\\The over-arching science question we need to address is how much exploitation, disturbance or stress can a particular ecosystem withstand without the loss of resilience and a range of other ecosystem services and values. Frameworks are beginning to be developed though.
		\\Until our ability to predict shifts in resilience improves, management and policy should focus on insurance and capacity maintenance (reserve...).
		\\\textbf{Relevancy for PhD} : High. To be inspired for R-S framework, from theory to field sampling.
		\item \cite{Murray2018} : Role of satellite remote sensing in risk assessments of ecological collapse.  Satellite remote sensing can deliver ecologically relevant, long-term datasets suitable for analysing changes in ecosystem area, structure and function at the temporal and spatial scales relevant to risk assessment protocols. However, there is considerable uncertainty about how to select and effectively utilise remotely sensed variables for risk assessment. Here, we review the use of satellite remote sensing for assessing spatial and functional changes of ecosystems.
		\item \cite{van2017timing} : Protocol using an algorithm to measure the recovery time after a disturbance from sequential spatial data. The algorithm can be applied to both empirical, e.g. remotely sensed, and simulated spatial data. To be inspired for the PhD.
		\item \cite{sala1998fishing} : 2 alternate algal assemblage (Alt Stab State) fleshy erect algae communities vs sea urchin/coralline barrens mainly driven by abundance of sea urchins, controlled by predator themselves (over)fished. But surely not the only factor, others are discussed. In Med only though.
		\item \cite{rockstrom2009safe} : A Safe Operating Space for Humanity.
		\\Holocene : past 10'000 years of unusually stable environment.
		\\Human activities have reached a level that could damage the system that keep the Earth in the desirable Holocene-state. Without pressure forces, Holocene is expected to continue for at least several thousands of years.
		\\Define 9 Planet Boundaries : The safe operating space associated with the planet biophysical subsystems or processes.
		\\ Many subsystems of Earth reactions are non-linear, often abrupt around threshlods levels of certain key variables.
		\\Job of trying to quantify the safe limits outside of which the Earth system cannot continue to function as a stable Holocene-like state.
		\\Approach on 3 branches of scientific enquiries: scale of human actions in relation to earth to sustain it; Working on understanding Earth processes (including human); Research into resilience and links to complex dynamics living systems emphasing thresholds between states.
		\item \cite{ripple2019world} : World Scientists Warning of a Climate Emergency.
		\\ Scientists have a moral obligation to clearly warn humanity of any catastrophic threat and to “tell it like it is.” On the basis of this obligation and the graphical indicators presented below, we declared with more than 11,000 scientist signatories from around the world, clearly and unequivocally that planet Earth is facing a climate emergency.
		\\ We suggest six critical and interrelated steps (in no particular order) that governments, businesses, and the rest of humanity can take to lessen the worst effects of climate change.
		\\ Actions on 6 areas : Energy, Short-lived pollutants, Nature, Food, Economy and population.
		\\We must protect and restore Earth’s
		ecosystems. Phytoplankton, coral reefs,
		forests, savannas, grasslands, wetlands,
		peatlands, soils, mangroves, and sea
		grasses contribute greatly to sequestration
		of atmospheric CO2
		\\\textbf{Relevancy for PhD} : High as set framework of the need of finding ecological tipping points to ensure ecosystem services. To link with paper on the safe operating space for humanity and IPCC reports. 


		\subsection{Kelp Forests}
		\item \cite{filbee2014sea} : Sea urchin barrens as alternative stable states of collapsed kelp ecosystems.
		\\ Evidence of transitions between barriers and kelp beds in several regions.
		\\Review of theoretical concepts of Regime Shifts, ecologically oriented, nice!
		\\Good figure of Ball-in-cup explaining shifts in different locations.
		\\Types of feedback mechanisms that stabilize community assemblage: reduce kelp recruitment and allow sea urchins at high densities.
		\\The trend of increasing marine species richness and ecosystem complexity from the Arctic to the tropcis (Gray2001) may explain this discrepancy because the more simplified food webs in temperate ecosystem may collapse more readily.
		\\\textbf{Relevancy for PhD} : Very high, evidence to support Urchin barren as ASS. Not the kind of transitions we might focus to as drivers are mainly biological one (grazing and predation on fish eat urchin larvae).	
		\item \cite{filbee2018rise} : Rise of Turfs: A new battlefront for globally declining kelp forests.
		\\Kelp forests provide valuable services along 25$\%$ of the world's coastlines.
		\\ Role and importance of kelp forests in development of humanity, review of different services.
		\\ Seagrass and seaweed beds are the third most productive system globally providing services of roughly US3,000,000 per km of coastline per year (Costanza2014).
		\\Shift to turfs is a new battlefront as kelp forests move away from traditional urchin grazing dynamics to climate and nutrient driven replacemennt by turf $\rightarrow$ kind of shift of interest in PhD because focussed on light as climate driver.
		\\Examine the feedback mechanisms that prevent recovery of kelp forests.
		\\Turf=algae that provide little to no 3D seascape structure compared with kelp. No commonly accepted definition of turf.
		\\Key questions: What are the main drivers of shifts to turfs? What feedback mechanisms are maintaining them? How permanent are they? What strategies to adopt to address the problem?
		\\Suite of stressors and env. changes can lead to kelp loss and shift to turf (Strain2014).
		\\Most of the collapsed kelp forests have been located in warming hotspots or near the edges of their distribution, where they likely are less resilient to additional perturbation. $\rightarrow$ In the project : is the light a parameter influencing resilience of kelp forest? If yes, promoting a certain light environment will make KF more resilient to add perturbation.
		\\Faster canopy recovery at cooler southern locations compared to warmer northern location (WA) $\rightarrow$ EWS!
		\\Eutrophisation reduces light penetration in coastal waters and can favour the persistence of turf, which have high growth rates and rapid nutrient uptake rates compared with largerm canopy-forming algae (Gorman2009) $\rightarrow$ Path to investigate : Strategy to light field adaptation (game theory, energy budget).
		\\In the N Med Sea, Sweden, Denmark and South Oz, disappearance of canopy kelps and other macroalgae was largely attributed to increases in coastal nutrients and sediment loading $\rightarrow$ relevancy on working on light.
		\\Turf can be not considered as true ASS, but what matter is that the key driver of kelp loss are likely to intensify, involving feedback mechanisms difficult to reverse. Consequences will be serious on timescales relevant to humans.
		\\No kelp-to-turfs have been documented in cool-water upwelling zones: N/S America west coast or Southern Africa.
		\\Far more attention and research has been given to loss of coral reefs compared with loss of kelp forests and seagrass beds.
		\\\textbf{Relevancy for PhD} : Very high, discussion about turf as alternative stable state of Kelp forests.
		\item \cite{krumhansl2016global} : Global patterms of kelp forest change over the past half-century.
		\\38$\%$ of the world's kelp forests have been in decline over the past five decades.
		\\Although interactions between local, regional and global processes have produced complex responses in terms of the direction and ultimate drivers of kelp forest change.
		\\\textbf{Relevancy for PhD} : Very high, global situation of Kelp worldwide, to cite.
		\item \cite{tait2019giant} : Giant Kelp forests at critical light thresholds show compromised ecological resilience to environmental and biological drivers.
		\\ Thresholds of urchin density and Temp $\rightarrow$ well established but Turbidity rarely considered.
		\\Kelps and fucoids responded to similar light thresholds. Kelp more vulnerable to physical disruption by sed (change of K$_d$) whereas fucoids more vulnerable to declining light availability (Ebed).
		\\Intro on kelp global dynamics. Review of different drivers of collapse (nuts, overfishing, storms and waves, sedimentation, heat waves). Role o flight for kelp.
		\\ Critics on 1\% from Irr surface : Sporophytes and gametophytes have different light requirement.
		\\ Light energy budget are universally important for habitat seaweed $\rightarrow$ potential fundamental basis for identifying thresholds (vs observed variation of driving parameters, difficult to provide generalities for management).
		\\Use data of PAR (K and Ebed) over 2 years in \textit{M.pyrifera} forest in South NZ. Light sensors deployed at 2 depths (6-7m and 9-10m).
		\\Light thresholds for gametophytes = 0.4 mol.m-2.d-1, sporophytes 1.0 and 1.5 for a saturated growth.
		\\ Gradient of $\downarrow$ light availability are significantly correlated with $\downarrow$ richness and abundance of habitat-forming seaweed.
		\\Biblio on change of land use to $\downarrow$ light delivery to benthos.
		\\ Buffering effect of disturbances with depth $\rightarrow$ $\downarrow$ light means compression of depths range and $\downarrow$ resilience. 
		
		\item \cite{desmond2015light} : Effects of light limitation within kelp forests communities structure in NZ.
		\\ Light penetration controls macroalgal depth distribution if suitable substrate and lack of significant grazing pressure is assumed.
		\\Globally macroalgal distribution has been estimated to be limited by light in 34-54\%.
		\\Light limitation : potential to $\downarrow$ depth distribution of species, $\downarrow$ growth rate, $\downarrow$ community complexity $\rightarrow$ $\downarrow$ primary productivity, habitat availability as well as ecosystem resilience to stress.
		\\Growing concern to loss of macroalgal dominated habitats worldwide.
		\\Compared 2 reefs : intact native in Stewart Island vs modified in east Otago.
		\\Possible that decrease in light availability may have aided the success of \textit{Undaria pinnatifida} by excluding native species with less efficient light harvesting abilities and lower photosynthetic rates $\rightarrow$ link to Masters thesis.
		\\Surprising how little is known about the temporal and spatial variability of light within the coastal ecosystems.
		
		\item \cite{wernberg2010decreasing} :	Soutwest coast of Oz taken as a case of study of future global conditions (warmer and lower nuts levels), place to experiment links between physiological and ecological performances of Kelp beds in south hemisphere. What would be the cost of living in a warmer ocean climate?
		\\Disturbance experiment in 2 groupes (2 stations each) Warm/Cool.
		\\ >19 degC is a threshold for \textit{Ecklonia radiata} about growth and productivity. 
		\\ Hypothesis that ocean climate and $\uparrow$ of severity of perturbations $\downarrow$ ecological performance (recruitment, growth) of kelp recruits and recovery of adult canopy was supported.
		\\Physiological cost of Thermal stress bring closer to threshold where failure to recover is more likely.
		\\Shift of strategy in ecological interaction between adult and recruit kelps from - (competition) to + (facilitation) under warmer conditions $\rightarrow$ mediate the erosion of resilience.
		\\Higher temperature is unlikely to pose physiological problems for most kelp beds population BUT reduced resilience can as it will exacerbate the effects of other stressors.
		
		\item \cite{wernberg2005effect} : Wave exposure (WE) effects on \textit{Ecklonia radiata} in SW Oz.
		\\WE is probably the most commonly identified cause of morphological variation in macroalgae.
		\\ Only little evidence (No statistical difference, only trends) that WE has a consistent effect on morphological variation in \textit{E.radiata} in SW Oz.
		\\ WE is a very complex factor with many attributes (speed, acceleration, lift, period, duration and direction).
		\\ Substratum can break before holdfasts.
		\\ $\rightarrow$ To read more about effect of wave exposure on kelp species.
		
		\item \cite{wernberg2016} : Regime shift of Kelp Forest in WA due tu increase of SST. Heat Wave of 2011 as disturbance that made crossing a thermal tipping point. No remote sensing sample of kelp. $\rightarrow$ Could be of inspiring for my PhD, combined effects of increase of turbidity and Temperature.
		\item \cite{thomsen2019local} : Role of marine heatwave in the local extinction of Bull Kelp in South Island NZ. Replacement by introduced species (Undaria) and grassy seaweed (Ulva) --> Could be interesting to link with my masters degree.		
		\item \cite{schiel2019kaikoura} : Effects of Kaikoura EQ on benthic communities. "New Intertidal zone is steeper and have less overall area. Little evidence that the formerly abundant species will be able to move simply move down because of the general unavalaibility of suitable rocky reef. Increase of sediment load as well. Loss of connectivity.
		"this situation presents the reverse of modelled simulations of ‘tipping points’ from a desired state into some alternative undesirable state of the ecosystem" 
		\item \cite{leleu2012mapping} : 30-year difference study of 1st NZ No-Take reserve, show spatial evidence of recovery of Kelp Forest over Barren Ground. Role of fishing highlighted but others factors also evoked.

		
		
		\subsection{Studies on other systems}
		
		\item \cite{van2017vegetation} : 
		Mathematical modelling has revealed that generic indicators may exist for a broad class of systems and	that can serve as early warnings to inform whether resilience is in decline. These indicators are based on the phenomenon of ‘critical slowing down’, which means that the time needed for a system to recover from a disturbance lengthens when the level of stress applied on the system increases. Temporal and spatial statistical signatures of slowing down have been inferred indirectly from fluctuations and correlations in system states and highly controlled experiments.
		\\However, direct measurements of the recovery rates that test this theory in the complex setting of real-world ecosystems.
		\\ Here we bridge this gap between theory and application by examining if critical slowing down can be observed along gradients in environmental stress in tidal marsh ecosystems.
		\\ Use of remotely sensed imagery (Time series of images) to estimate recovery rate as resilience indicator.
		\\\textbf{Relevancy for PhD} : Very high, to be inspired when use drone footage.
		
		\subsection{Deep Water Kelp}
		 \item \cite{nelson2015beyond} : Summary of MPO living deep in the euphotic zones around NZ. Evokes the role of deeper communities in the resilience of coastal communities (to read Graham 2017). $\rightarrow$ Role/evolution of Seabed light for this deep communities, great potential to link PhD project with this study.
		 \item \cite{graham2007deep} : Deep water kelp refugia in tropical waters = habitable areas below uninhabitable surface waters. Kelp requires substrate, high nut (cool water) + min annual irradiance dose ($>50E.m^{-2}$). Model combining sat PAR, Kd, bathy and estimated mixed layer depth give location of illuminated substrate. Existence of Deep Water refuge for tropical kelp helps understand origins in the present day distribution of surface kelp and buffer kelp system from climate phenomena (ENSO) -> factor of resilience! Mixed effect of $\uparrow$ SST on limits of refugia :
		 \\ $\downarrow$ surface prod $\uparrow$ optical clarity $\uparrow$ lower limit
		 \\ $\downarrow$ mixed layer $\downarrow$ upper limit of refuge 
		 \\ Vertical extent of mixed layer estimated by depth of the seasonal thermocline, vertical Temperature profiles data from WOD2001 (interpolated and used to estimate it, kelp more vulnerable to nutrient limit than Temp stress)
		  \item \cite{runcie2008situ} : Photosynthetic rates of macro algae at their lower depth limit. Strategies employed by deep water macroalgae (reviewed by Raven 2000). In situ irradiance measurements. Seems plausible that deep water individuals are derived from shallow water individuals. No kelp in this study though. Medium relevance of PhD.
	\end{itemize}
	
	
	\section{On NZ Environment generalities}
	\begin{itemize}
		\item \cite{macdiarmid2013new} :EEZ + extended continental shelf = 21x NZ land area, 1.7\% of worlds ocean.
		\\Surface dissolved CO2 indicates that NZ EEZ CO2 sink $\approx$ 5\% of global ocean CO2 uptake, larger than NZ forests.
		\\Good review of NZ numbers to know about NZ etent marine ecosystems.
		\\Different classifications of NZ marine environments.
		\\At least 60 distinct ecosystems can be identified.
		\\ Only 15 \% of total area has been swath-mapped to standar necessary to map benthic habitats.
		\\ Ecosystem services : Direct or indirect benefits that humans receive or value from natural or semi-natural habitats (Costanza 1997).
		\\3 broad groups of services : Regulatory, Provisioning and Non-consuming services.
		\\ In NZ : 12 regulatory, 5 provisioning and 9 non-consuming.  
		
	\end{itemize}
	
	\section{On Light and Remote Sensing}
	
		\begin{itemize}
			
		\item \cite{roberts2018tidal} : Tidal Modulation of Seabed light and its implications for benthic algae.
		\\ Demonstrated here for the first time, neglecting tidal effects on seabed light is likely to result in erroneous estimates (and, for many sites, underestimation) of subtidal benthic productivity.
		\\ Not a direct link though between seabed light and photosynthesis/Productivity.
		\\ Take tide into account in estimating Ebed. PAR, KPAR Not from Satellite Data though. PAR from solar angle. KPAR mean value per day depending on tidal range.
		\\ Comparison to measurements. 
		\\\textbf{Relevancy for PhD} : High as it will feed the discussion about using tide in generation of Ebed data through Bathy file.
		
		\item \cite{seers2015spatio} :Spatio-Temporal patterns in coastal turbidity-long-term trends and drivers of variation across an estuarine-open coast gradient.
		\\Analysed 22 years of coastal turbidity data along an estuarine to open-coast gradient in northern New Zealand
		\\ No region-wide changes in turbidity are evident, despite implementation of land management regulations.
		\\STL method on trends
		\\To read and sum up more in detail
		\\\textbf{Relevancy for PhD} : Fair. Origin and history of turbidity in NZ of interest. Use of STL trends as well.
		
		\item \cite{kirk1994} : Good definitions on optical parameters, see LateX file summary of the book.
		\item \cite{magno2019model} : Model for deriving benthic irradiance in the GBR from MODIS
		\\ Intro to be inspired for PhD. Context and explanation of measuring/calculating Ebed (bPAR in the paper).
		\\ Benthic ligth availability remains poorly understood because data sets that could provide th emuch needed info are either insufficient or lacking.
		\\ Usable Solar Radiation (USR), a spectrally integrated broadband (400-560nm) term has recently been proposed as a promising alternative to PAR as its diffuse attenuation coefficient Kd exhibit less depth dependence than Kd PAR.
		\\Type II linear regression analysis was used to compare satellite-driven and in-situ measure of Ebed (log 10 transformed data).
		\\ One of the possible source of error of our bethic irradiance model could be driven by changes in water-column depth dut to tidal fluctuation $\rightarrow$ conduct a Ebed model run using in-situ depth pressure (from sensor data).
		\\Use radiative transfer simulations (Hydrolight) vs modeled Ebed values. Strong agreement between concurrent satellite-derived and in-situ Ebed data pairs at 4 test sites.
		\\ For more inshore waters with IOPs varying at shorter time-scales or higher frequencies, it becomes more challenging to temporally extrapolate IOP over the whole day.
		\\Published ecologically-relevant min values of Irradiance required to maintain autotrophic health may vary between 0.4 to 5.1  mol.m-2.d-1, ref Gattuso2006 "Light availability in the coastal ocean...".
		\\ Upcoming Plankton, Aerosol, Cloud and ocean Ecosystem PACE mission scheduled 2022, to read about.
		\\Model limitation : Clear sky / Do not include diffuse sky irradiance / Contributions of wind-roughened sea surface to reflectance-transmittance across the interface / Limited in areas where tidal fluctuations are much higher than what they tried to address.
		\\ Resulting benthic light datasets will provide information needed to assess habitat quality for corals and seagrasses, specially maps of light thresholds.
		\\\textbf{Relevancy for PhD} : Really High. Methods to compare in-situ vs satellite derived Ebed to be inspired from. General context of role of using satellite in Ebed to be inspired from as well. Intro to Hydrolight as well?
	\end{itemize}

	\section{Solar Irradiance generalities (and SST)}
\begin{itemize}
	\item \cite{white1997response} : Response of Global Upper Ocean Temperature to changing Solar Irradiance.
	\\ Decadal and Interdecadal changes are ubiquitous in SST, rainfall, forest fires, cyclones.
	\\ But limited geo-cover of records $\rightarrow$ unreliable indicators for solar related climate change.
	\\ Heat budget of oceanic response to changing solar irradiance : Where is the anomalous heat from changing solar irradiance stored?
	\\ Use of 2 equations : global average anomalous heat budget for the upper ocean and a modified form of Stefan Boltzmann law (radiation balance). Relation between $\delta$Solar irradiance and $\delta$Temperature. $\rightarrow$ Could be of relevance for PhD if SST taken into account.
	\\ Solar-related signals in upper ocean Temperature penetrate to 80-160m depth $\rightarrow$ Anomalous heat is stored in the upper 100m, heat balance maintained by heat loss to the atmosphere no to deep ocean.
	\\Global average SST anomalies can be explained by changing solar irradiance at Sea Surface. $\rightarrow$ Is it the case in coastalNZ ?
	\\ Estimation of an ocean's climate sensitivity to changing solar insolation from heat budget and radiation balance equations.
	\\ Response of upper ocean Temperature to changing solar irradiance is not uniform over the global ocean.
	\\Critics : Solar insolation is not the only mechanism for explaining how solar irradiance can induce changes in global-average upper ocean Temp.
	\\\textbf{Relevancy for PhD} : Methods of analysing Temporal series. Perspectives to use when dealing with SST. What would be the ocean climate sensitivity to changing solar irradiance in NZ and what could be the implication for benthic communities? But first need to identify trends in sea surface irradiance. If so, to what extent? Would it affect physiology plant (less/more light)? And what implication with the depth?
	\item \cite{sun2007solar} : Solar Radiation model with Fourier Transform approach.
	\\Talk about deterministic vs stochastic approach to estimate irradiation.
	\\ When examining irradiation itself : time series analysis are traditionally done of irradiation records. Data are decomposed into identifiable seasonal trends of daily means and variance along with dynamic, unpredictable random portion called "noise".
	\\Revue of different model of irradiation.
	\\Detrend signal from seasonalities, substracting principal sinusoids. Get only noise signals at the end.
	\\Fast Fourier Transform on noise signal of every year.
	\\ Their goal is to generate annual DOI values that are statistically similar to those of measured physical data. Succeeded.
	\\\textbf{Relevancy for PhD} : Low. Work on noise from temporal series could be of interest though.
\end{itemize}

	\section{Mathematical and Statistical analysis}
\begin{itemize}
	
		\item \cite{abraham2003unsupervised} : Unsupervised Curve Clustering using B-Splines.
		\\ Although data are gathered as finite vector and may contain measurement
		errors, the functional form have to be taken into account. We propose a clustering procedure of
		such data emphasizing the functional nature of the objects. The new clustering method consists of
		two stages: fitting the functional data by B-splines and partitioning the estimated model coefficients
		using a k-means algorithm.
	\\\textbf{Relevancy for PhD} : High, methodology of clustering trend curves.
	
	\item \cite{qian2016environmental} : Environmental and Ecological Statistics with R. Tests and inference Statistics. Linear and non linear regression models, classification, Generalised Linear Model (GLM), model checking (bootstrap), Multilevel regression (ANOVA). Application in R.
	\\\textbf{Relevancy for PhD} : High. One of these bibles.
	\item \cite{cleveland1990stl} : STL Method.
	\\ Design of STL and the choice of parameters in practice are based on an understanding of which part of the variation in a time series becomes the seasonal components and which part becomes the trend components. This understanding comes from the eigenvalues and frequency response analyses (section 5).
	\\ STL has 6 parameters : $n_p$ : number of observation in each cycle of the seasonal component, $n_i$ : number of passes through the inner loop, $n_o$ : number of robustness iteration of the outer loop, $n_l$ : the smoothing parameter for the low filter, $n_t$ : the smoothing parameter for the trend component, $n_s$ : smoothing parameter for the seasonal component.
	\\ 2 roles of the trend component plays in helping to estimate the seasonal comp :
	\\- To remove the persistent, long term variation (that cause the linear increase of CO2 example). If not remove, it will distor the seasonal comp. The presence of such behaviour is what prevent us to simply apply an ordinary filter to the data that passes in bands centered at the fundamental seasonal frequency and its harmonics.
	\\- Also plays a role in the robustness iteration.
	\\ $n_{(t)}$ too large : allow major low-frequency efects to go to the remainder. This method wants to give reduced weight only to extreme and transient (not permanent) behaviour and not to peaks and troughs of major slow oscillations. For this purpose we want $n_{(t)}$ to be small.
	\\ But not too small because we don't want trend and seasonal component to compete for the variation in the data.
	\\ Rules of :
\begin{equation}
	\label{eqSTL}
n_t \geqslant \frac{1.5n_{(p)}}{1-1.5n_{(s)}^{-1}}
\end{equation}
	\item \cite{verbesselt2010detecting} : Detecting Trend and seasonal changes in satellite image time series. BFAST procedure.
	\\ Generic approach for detection and characterisation o fchange in time series : Breaks For Additive Seasonal and Trend (BFAST).
	\\ Same procedure of building an additive decomposition model as STL (Trend, Seasonal and Remainder) : $Y_t=T_t+S_t+e_t$.
	\\ But it is assumed that $T_t$ is piecewise linear, with break points $t_1\star,...t_n\star$. $T_t=\alpha_j + \beta_jt$, with $\alpha$ and $\beta$ the intercept and slopes that can be used to derive the magnitude and direction of the abrupt change, referred as magnitude.
	\\Magnitude = ($\alpha_{j-1} - \alpha_j$)+($\beta_{j-1}- \beta_j$)t
	\\BFAST detects multiple changes in Time Series : dates and number of change occuring within seasonal and trend components, extract the magnitude and the direction of change.
	\\Change in trend component indicates gradual and abrupt change whereas Change in seasonal component indicates phenological changes.
	\\Approach can be applied to T-S without the need to select specific land cover types, select a reference point, set a threshold or define a change trajectory.
	\\\textbf{Relevancy for PhD} : High. Can be better than STL to sort out trends curves as they are piecewise linear. Number and date of thresholds can be used to classify/clusterise pixels.
	\item \cite{jacques2014functional} : Functional Data Clustering : a survey.
	\\ Most approaches used for clustering functional data are based on the following 3 methodologies : Dimension reduction before Clustering (2-stage approaches) / Non parametric methods using specific distance or dissimilarities between curves / Model-based clustering methods.
	\\ Aim of the cluster analysis is to build homogeneous groups (clusters) of observations representing realisations of some random variable X.
	\\Clustering is often used as a preliminary step for data exploration to identify particular patterns in data.
	\\ Mathematical demonstration of FPCA and eigenvalues problem.
	\\First step when working with functional data is to reconstruct the functional form of data.
	\\ Second important step is data registration : centring and scaling curves to eliminate phase and amplitude variations in curves dataset. For \cite{jacques2014functional} it is not necessary to do so as amplitude and phase variability of curves can be interesting elements to define clusters (cf Canadian weather dataset). Absence of cluster after registration often observed.
	\\\textbf{2-stage approaches} : first step (reducing dim or filtering) by approximating curves into a first basis of function (Spline basis most common choice) or FPCA. Second step is the use of clustering algorithms on coefficients in a basis of functions or on their first principal components scores.
	\\\textbf{Non-parametric approaches} : 2 methods. First is applying non parametric clustering techniques (K-means or Hierarchical clustering) with specific distances or dissimilarities. Second is proposing new heuristic and geo criteria to cluster.
	\\ Use of raw-data clustering is probably the worst choice since it does not take into account the "time dependent" structure of data (inherent to funct data).
	\\2-stage methods consider it since first step is approximating curves into a finite basis of function. Main weakness is that the filtering step is done previously to clustering and the independently of the goal of clustering.
	\\ Non-parametric methods have advantage of their simplicity, easy to understand and to implement. But also weakness as complex cluster structure can not be efficiently modelled by this approach.
	\\ To their opinion : best methods are model-based clustering coz take into account the functional nature of datam perform simultaneously dimensionality reduction and clustering and allow model complex covariance structure.
	\\Comparison of different packages/methods of clustering in respect to Correct Classification Rates (CGR).
	\\\textbf{Relevancy for PhD} : High. As describe aims, processes and different methods to clusterise functional data.
	
\end{itemize}

	\newpage
	\bibliographystyle{apalike} %plain, acm, alpha, apalike, ieeetr, siam, unsrt, ...
	\bibliography{biblio}
	
	
\end{document}